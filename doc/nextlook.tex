\documentclass[a4paper,11pt]{report}
\usepackage[T1]{fontenc}
\usepackage[utf8]{inputenc}
\usepackage{lmodern}
\usepackage{listings}
\usepackage[hidelinks]{hyperref}


\title{NeXtlook}
\author{Darryn Jordan}

\begin{document}

\maketitle
\tableofcontents

\begin{abstract}
NeXtlook is a quicklook processor for generating range-time-intensity and range-Doppler plots of NeXtRAD datasets. Multithreading is supported for pulse-compression, but Doppler-processing is currently only supported when using a single thread.
\end{abstract}

\chapter{Installation}
\textit{This chapter documents the installation process for the NeXtlook processor.}
\section{Prerequisites}
\subsection{FFTw}
Download the latest version of FFTw from their website ensuring that the version number is 3.3.6 or higher. This version introduced thread-safe planners which are vital. 
Extract the files in an arbitrary location, enter the extracted folder and run the following commands,
\begin{lstlisting}
./configure --enable-threads
make -j
sudo make install
\end{lstlisting}
If any problems occur I suggest using the following command to perform a clean before retrying,
\begin{lstlisting}
make distclean
\end{lstlisting}
The extracted folder can now be deleted.
\subsection{Misc.}
\begin{lstlisting}
sudo apt install cmake
sudo apt install git
sudo apt install libopencv-dev
sudo apt install libboost-all-dev
\end{lstlisting}

\section{NeXtlook}
At the time of writing, the NeXtlook source code is located at \url{https://github.com/darrynjordan/nextlook}. The source can be downloaded as a zip folder or cloned from the repo,
\begin{lstlisting}
git clone https://github.com/darrynjordan/nextlook.git
\end{lstlisting}
Within the cloned folder, create a new folder to hold the built executable and run Cmake,
\begin{lstlisting}
mkdir build
cd build
cmake ..
\end{lstlisting}
If no errors have occurred, the executable can be compiled and run as follows,
\begin{lstlisting}
make -j
./nextlook
\end{lstlisting}

\chapter{Usage}
\section{Experiment Configuration}
On first run, the program will probably complain about not finding the binary dataset. Do not panic. Within the root folder there should be a textfile called \textbf{experiment.ini} which contains all experiment parameters. This is where the path to the binary dataset and reference waveform, and other options must be configured.

\subsection{Example Parameter File}
\begin{lstlisting}
[config]
debug_mode = false

[dataset]
data_filename = ../12_12_2016_14_48_59_adc1data.dat
n_cmplx_samples_range_line = 2048
n_range_lines = 30000
ref_filename = ../Ref_PL_Xband_MPALNA_3us.dat
n_cmplx_samples_ref = 297
n_cmplx_samples_padded = 2048;

[processing]
n_threads = 1;
doppler_enabled = true
doppler_cpi = 256
range_window = UNIFORM
doppler_window = UNIFORM

[visualisation]
update_rate = 1024
doppler_colour_map = 1
rti_colour_map = 2
histogram_equalization = 1
slow = 0
threshold = 50
\end{lstlisting}

\subsection{Parameter Description}
\subsubsection{config}
\begin{itemize}
  \item \textbf{debug-mode} provides additional information through the command line during processing.
\end{itemize}
\subsubsection{dataset}
\begin{itemize}
  \item \textbf{data-filename} the path to the binary dataset to be processed. 
  \item \textbf{n-cmplx-samples-range-line} number of complex samples in each range line.
  \item \textbf{n-range-lines} total number of range lines to process.
  \item \textbf{ref-filename} the path to the binary reference to be processed. 
  \item \textbf{n-cmplx-samples-ref} the number of complex samples in the reference waveform.
  \item \textbf{n-cmplx-samples-padded} determines the number of complex samples each range line must be zero padded to.
\end{itemize}
\subsubsection{processing}
\begin{itemize}
  \item \textbf{n-threads} the number of threads which will be launched for pulse compression.
  \item \textbf{doppler-enabled} toggles Doppler processing.
  \item \textbf{doppler-cpi} number of range bins used for each Doppler plot.
  \item \textbf{range-window} tapering function applied to each range line. 
  \item \textbf{doppler-window} tapering function applied to each Doppler line.
\end{itemize}
\subsubsection{visualisation}
\begin{itemize}
  \item \textbf{update-rate} number of range lines to be processed before a plot update occurs.
  \item \textbf{doppler-colour-map} [0 - 11] index of colour map to be applied to range-Doppler plot. 
  \item \textbf{rti-colour-map} [0 - 11] index of colour map to be applied to RTI plot. 
  \item \textbf{histogram-equalization} [0 - 1] toggles histogram equalisation for both plots. 
  \item \textbf{slow} [0 - 500] number of milliseconds processing delay between each plot update. 
  \item \textbf{threshold} [0 - 255] apply threshold to both plots. 
\end{itemize}





\end{document}
